%%%%%%%%%%%%%%%%%%%%%%%%%%%%%%%%%%%%%%%%%%%%%%%%%%%%%%%%%%%%
% Template prepared by Dr Daniel Oi, CC BY-NC 4.0          %
%%%%%%%%%%%%%%%%Do Not Alter The Preamble%%%%%%%%%%%%%%%%%%%
\documentclass[aps,pra,a4paper,nofootinbib,preprint,12pt]{revtex4-1} % Uses the APS RevTeX document class
\usepackage{latexsym,graphicx,amsmath,amsfonts,amssymb,color,geometry}
\geometry{a4paper, portrait, hmargin=2cm, vmargin={2.5cm, 3cm}}
\usepackage{palatino} %Slightly nicer than TMR
\def\bibsection{\section*{\refname}} %Sorts out reference section and gets rid of horizontal line
\usepackage{titlesec}
\titleformat*{\section}{\Large\bfseries}
\titleformat*{\subsection}{\large\bfseries}
\titleformat*{\subsubsection}{\normalsize\bfseries\itshape}
\titleformat*{\paragraph}{\large\bfseries}
\titleformat*{\subparagraph}{\large\bfseries}
% \usepackage{sfmath}
% \renewcommand{\rmdefault}{phv}
% \renewcommand{\sfdefault}{phv}
%%%%%%%%%%%%%%%%%%% Preamble Ends Here %%%%%%%%%%%%%%%%%%%%%%
% Insert any other packages or definitions you want here.







%%%%%%% Fill in your details below %%%%%%
\newcommand{\projecttitle}{Experimental Technologies for Magneto-Optical Trapping}
\newcommand{\studentname}{Stuart Smyth}
\newcommand{\regnumber}{201527054}
\newcommand{\degree}{MPhys}
\newcommand{\primarysup}{Dr Oliver Burrow}
\newcommand{\secondsup}{Dr Paul Griffin}%Comment out if needed
%%%%%%%%%%%%%%%%%%%%%%%%%%%%%%%%%%%%%%%%

%%%%% Do Not Alter %%%%%%%%%%%%
\usepackage{fancyhdr}
\pagestyle{fancy}
\setlength{\headheight}{14pt}
\footskip = 45pt
\fancyhf{}
\lhead{\projecttitle} %If title is long, insert "\small", "\footnotesize", "scriptsize", or "\tiny" in front of "\projecttitle
%\rhead{} %Optional
\lfoot{PH450 Project Report}
\cfoot{Student: \regnumber}
\rfoot{\thepage}
%%%%%%%%%%%%%%%%%%%%%%%%%%%%%%%

%The document starts here
\begin{document}

% This section creates the cover page

\begin{figure}
\includegraphics[width=\textwidth]{ScienceLogo.png}
\end{figure}

\title{PH450 Project Report 2016-17\\ \vspace{1cm}
{\huge \projecttitle} %If title is too long, use \LARGE, \Large, or large instead of \huge
\\
\vspace{1cm}
{\footnotesize Submitted in partial fulfilment for the degree of \degree}}

\author{\studentname\\
Registration No.: \regnumber}
\affiliation{SUPA Department of Physics, University of Strathclyde, Glasgow G4 0NG}

\vspace{1cm}

\author{\primarysup}
\affiliation{Primary Supervisor}

\vspace{1cm}

\author{\secondsup} % Comment out these two lines if not required
\affiliation{Secondary Supervisor}

%Use if required
%\author{\thirdsup}
%\affiliation{Tertiary Supervisor}

\vspace{1cm}

\date{\today}

\pagenumbering{gobble}

\maketitle % This ends the title page section


\pagenumbering{roman} % Use Roman numerals for the front matter pages

\section*{Abstract}
\addcontentsline{toc}{section}{Abstract}
%%%%%%%%%%%%%%%%%%%%%%%%%%%%%%%%%%%%%%%%%%%%%%%%%%%%%%%%%%%%%
% Notes: Change text in red as appropriate to your report.  %
% Remove the "\textcolor{red}{}" parts so that final output %
% text is black.                                            %
%%%%%%%%%%%%%%%%%%%%%%%%%%%%%%%%%%%%%%%%%%%%%%%%%%%%%%%%%%%%%


\textcolor{red}{In a few sentences, concisely summarise your project and results. Indicate what research problem was addressed, the methods used, and the main conclusions.} 

\newpage
\section*{Preface}
\addcontentsline{toc}{section}{Preface}

\textcolor{red}{State explicitly how your dissertation relies on the work of others, highlighting the portions that you claim to be your own original work. E.g. ``The results presented in chapter 3 rely upon a simulation data provided by the research group. The data analysis is entirely my own work. The analysis in chapter 4 was performed in conjunction with my supervisor\ldots'' etc.  Without clarifying statements, it will be assumed that your thesis is a review article with no original content. Be sure to claim only your own work, any evidence to the contrary may leave you susceptible to charges of plagiarism.}

\newpage
\section*{Acknowledgements}
\addcontentsline{toc}{section}{Acknowledgements}

\textcolor{red}{You may want to acknowledge people who have helped you in your project.}

%%%%%%%%%%%%%%%%%%%%%%%%%%%%%
% Do not alter this section %
\newpage
\tableofcontents % Creates a Table of Contents
\makeatletter
\let\toc@pre\relax
\let\toc@post\relax
\makeatother 

\newpage % Creates List of Figures
\listoffigures
\addcontentsline{toc}{section}{List of Figures}

\newpage % Creates List of Tables.Comment out both lines if no tables.
\listoftables
\addcontentsline{toc}{section}{List of Tables}

\clearpage
%                           %
%%%%%%%%%%%%%%%%%%%%%%%%%%%%%


%%%%% Main Report Starts Here %%%%%

\newpage
\pagenumbering{arabic}
\section{Introduction}

\textcolor{red}{The introduction is where you should outline your project and your report. You should include background, context, previous work, the projects goals and aims, summary of methods, results, and ``take home message''~\footnote{This is a footnote.}.}

\subsection{\textcolor{red}{This is a subsection}}

\textcolor{red}{You can cite references like Ref.~\cite{smith2001} or like Refs.~\cite{smith2002,smith2003,smith2004,smith2005,smith2006,smith2007,smith2008,smith2009,smith2010,smith2011,smith2012} or like Refs.~\cite{smith2013,smith2014}.}

\subsubsection{\textcolor{red}{This is Sub-subsection}}
\label{sec:subssubsectionref} % This is how you create a section label that can refer to in the text later on.

\textcolor{red}{You can insert figures by using the appropriate {\LaTeX} commands. You can reference figures, e.g. Fig.~\ref{fig:example1}, by calling their labels. You can refer to specific sections of your report like Sec.~\ref{sec:subssubsectionref}.}

\begin{figure}[b]
\includegraphics[width=0.5\textwidth]{example.png}
\caption[\textcolor{red}{Short Caption for List of Figures}]{\textcolor{red}{This is a caption. Use captions effectively to provide details about the figure. Include information such as parameters, assumptions, models used, sample preparations, etc. Make sure figures are a good size.}}
\label{fig:example1} % This creates a figure label for cross-referencing in the text
\end{figure}

\textcolor{red}{Add equations like,}
\begin{equation}
\Delta T= \int_{t=0}^\infty ds f(s) e^{-i \frac{\hbar}{2\pi}\hat{K}}
\label{eq:example1} % This creates an equation label for cross-referencing in the text
\end{equation}
\textcolor{red}{and reference them like Eq.~\ref{eq:example1}.}

\newpage
\section{\textcolor{red}{Another section}}

\begin{table}[b]
\begin{center}
 \begin{tabular}{||c c c c||} 
 \hline
 Col1 & Col2 & Col2 & Col3 \\ [0.5ex] 
 \hline\hline
 1 & 6 & 87837 & 787 \\ 
 \hline
 2 & 7 & 78 & 5415 \\
 \hline
 3 & 545 & 778 & 7507 \\
 \hline
 4 & 545 & 18744 & 7560 \\
 \hline
 5 & 88 & 788 & 6344 \\ [1ex] 
 \hline
\end{tabular}
\end{center}
\caption[\textcolor{red}{Short Table Caption}]{\textcolor{red}{This is a table caption}}
\label{table:example1} % This creates a table label for cross-referencing in the text
\end{table}

\begin{figure}[b]
\includegraphics[width=0.5\textwidth]{example.png}
\caption[\textcolor{red}{Another Short Caption}]{\textcolor{red}{This is another caption.}}
\label{fig:example2}
\end{figure}

\newpage
\section{\textcolor{red}{Add Sections as Required}}

\newpage
\section{\textcolor{red}{Remember to Summarise and Discuss Results}}

\appendix

\newpage
\section{Appendix A: \textcolor{red}{First Appendix}}


\newpage
\section{Appendix B: \textcolor{red}{Second Appendix}}


\newpage

[\textcolor{red}{Ideally, you should use BibTeX to organize your references. These examples are simply to show formatting.}]

\begin{thebibliography}{99}

% These citations are included as examples. You should use a bib file to store your references and BibTeX to process them as part of the LaTex process.

\bibitem{smith2001}A. Smith, ``The unbearable lightness of seeing'', J. Phys. Q \textbf{23}, 3644 (2001)

\bibitem{smith2002}A. Smith, ``The unbearable lightness of sitting'', J. Phys. Q \textbf{24}, 12 (2002)

\bibitem{smith2003}A. Smith, ``The unbearable lightness of gardening'', J. Phys. Q \textbf{25}, 247 (2003)

\bibitem{smith2004}A. Smith, ``The unbearable lightness of rowing'', J. Phys. Q \textbf{26}, 9087 (2004)

\bibitem{smith2005}A. Smith, ``The unbearable lightness of flying'', J. Phys. Q \textbf{27}, 343 (2005)

\bibitem{smith2006}A. Smith, ``The unbearable lightness of cycling'', J. Phys. Q \textbf{28}, 2 (2006)

\bibitem{smith2007}A. Smith, ``The unbearable lightness of meeting'', J. Phys. Q \textbf{29}, 543 (2007)

\bibitem{smith2008}A. Smith, ``The unbearable lightness of tasting'', J. Phys. Q \textbf{30}, 2345 (2008)

\bibitem{smith2009}A. Smith, ``The unbearable lightness of smelling'', J. Phys. Q \textbf{31}, 1112 (2009)

\bibitem{smith2010}A. Smith, ``The unbearable lightness of touching'', J. Phys. Q \textbf{32}, 764 (2010)

\bibitem{smith2011}A. Smith, ``The unbearable lightness of jumping'', J. Phys. Q \textbf{33}, 243 (2011)

\bibitem{smith2012}A. Smith, ``The unbearable lightness of running'', J. Phys. Q \textbf{34}, 433 (2012)

\bibitem{smith2013}A. Smith, ``The unbearable lightness of skipping'', J. Phys. Q \textbf{35}, 256 (2013)

\bibitem{smith2014}A. Smith, ``The unbearable lightness of falling'', J. Phys. Q \textbf{36}, 666 (2014)


\end{thebibliography}

\end{document} 